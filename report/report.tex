% Graham Greving
% ggreving@ucsc.edu

% David Taylor
% damtaylo@ucsc.edu

\documentclass[10pt, letterpaper]{article}

\usepackage{fullpage}
\usepackage{hyperref}
\usepackage{listings}

\title{CodeChat}
\author{Graham Greving \\ David Taylor \\ Jake VanAdrighem}
\date{March 14th 2015}
\begin{document}

\maketitle

\section*{Introduction}

% Project intro CodeChat is a collaborative source-code editing environment. 
The project arose from the need to share code over IRC channels while working 
on group projects. Group projects require frequent working session. However, 
an academic setting often times means a lack of a shared workspace and 
working hours, as is the case in an office environment. This results in a 
need to have tools to enable remote collaboration. The solution to this 
problem was to share code over paste-bin and other such text sharing sites 
--- this leaves much to be desired. An ideal solution would be an environment 
where collaborating programmers can actively make changes to source code that 
would be reflected in real-time, and can be discussed using an IM platform. 
Ideally, this functionality would be in a unified platform, so these 
activities can be done seamlessly. This desire led to the idea of CodeChat.

% Languages intro When planning the project, we were faced with the choice of 
which programming language(s) to use. We wanted to use a language with good 
networking support and a good concurrency model. We also wanted a language 
that is simple and intuitively clear, even if that meant that the code was 
more verbose. These requirements led us to choose Go as the primary language 
for the project. After the system was implemented, we were curious how the 
developer and user experience would change if implemented in a language with 
an opposing paradigm to Go: this led us to try to implement select portions 
of the code in Python. This exposure to multiple languages gave certain 
insights into how these two languages relate to one another.

\section*{Architecture}

As this application requires connecting various users together, we figured 
that a client-server approach would be ideal. We also thought it would be 
best to implement the server with a modular and language-agnostic text 
protocol. While this may introduce added overhead in terms of performance, 
code size, and design/development time, the benefits of being able to create 
clients in a variety of languages, potentially implementing unique 
subsets of the server's commands, far outway any performance hits. The 
general architecture is a simple command protocol whereby a client can 
issue a command, and some optional payload, which the server then processes  
and writes back to all the other clients, as well as a status message to the 
issuing client.

\subsection*{Protocol}

This modularity allows the server to implement various commands and 
distribute the commands back to the connected clients. The clients, however 
can implement any particular subset of the server's commands, they only have 
to implement the bare minimum, connect and exit, to talk to the server. This 
means that while our server has the ability to support collaborative text 
editing, any client can simply implement a chat program with the same server, 
and ignore any unsupported commands from the server. Very cool.

To achieve language-agnosticity, we implemented a JSON protocol. We chose 
JSON because it is quickly (if not already) becoming a standard in server 
communications and most languages include libraries and packages for creating 
JSON objects. Additionally, since JSON is simply text, it can be very easily 
written to web sockets/connections in any language. I mentioned earlier the 
tradeoffs, this is because when you implement a client and a server in the 
same language, you can take advantage of expected language features on both 
sides of the connection. This is especially the case in a language like Go, 
whereby communication across the network can essentially be the same as 
communication across threads (with known/expected datatypes).

Currently implemented commands are: "connect", "exit", "update-file", 
"message".

\subsection*{Server}

The server is intended to be running on some network machine with a publicly 
available IP address, it can additionally be run locally --- but this inhibits 
the usefulness outside of debugging purposes. The program establishes a 
network connection and loops listening for incoming connections (as most 
servers do), when it recieves a connection, it spawns a separate goroutine 
(a lightweight thread) to handle this connections commands, then continues 
listening for more connections.

Writing back to clients is handled in a single goroutine which is executed 
before any incomming connections are established. This thread reads on an 
internal channel.

\subsection*{Client}

\section*{Languages}

\subsection*{Go}

\subsection*{Python}

\end{document}
