% Graham Greving
% ggreving@ucsc.edu

% David Taylor
% damtaylo@ucsc.edu

\documentclass[10pt, letterpaper]{article}

\usepackage{fullpage}
\usepackage{hyperref}
\usepackage{listings}

\title{CodeChat}
\author{Graham Greving \\ David Taylor \\ Jake VanAdrighem}
\date{March 14th 2015}
\begin{document}

\maketitle

\section*{Introduction}

% Project intro
CodeChat is a collaborative source-code editing environment. The project
arose from the need to share code over IRC channels while working on group 
projects. Group projects require frequent working session. However, an
academic setting often times means a lack of a shared workspace and
working hours, as is the case in an office environment. This results in
a need to have tools to enable remote collaboration.
The solution to this problem was to share code over paste-bin and other
such text sharing sites --- this leaves much to be desired. An ideal
solution would be an environment where collaborating programmers can
actively make changes to source code that would be reflected in
real-time, and can be discussed using an IM platform. Ideally, this
functionality would be in a unified platform, so these activities
can be done seamlessly. This desire led to the idea of CodeChat.

% Languages intro
When planning the project, we were faced with the choice of which
programming language(s) to use. We wanted to use a language with
good networking support and a good concurrency model. We also
wanted a language that is simple and intuitively clear, even if
that meant that the code was more verbose. These requirements led
us to choose Go as the primary language for the project. After
the system was implemented, we were curious how the developer and
user experience would change if implemented in a language with
an opposing paradigm to Go: this led us to try to implement
select portions of the code in Python. This exposure to multiple
languages gave certain insights into how these two languages relate
to one another.

\section*{Architecture}

\subsection*{Server}

\subsection*{Client}

\section*{Languages}

\subsection*{Go}

\subsection*{Python}

\end{document}
