% Graham Greving
% ggreving@ucsc.edu

% David Taylor
% damtaylo@ucsc.edu

% Jake VanAdrighem
% jvanadri@ucsc.edu

\documentclass[12pt, letterpaper]{article}

\usepackage{fullpage}
\usepackage{hyperref}
\usepackage{listings}
\usepackage{units}

\title{CodeChat}
\author{Graham Greving \\ David Taylor \\ Jake VanAdrighem}
\date{March 14th 2015}
\begin{document}

\maketitle

\section*{Introduction}

% Project intro

CodeChat is a collaborative source-code editing environment. The 
project arose from the need to share code over IRC channels while 
working on group projects. Group projects require frequent working 
sessions. However, an academic setting often times means a lack of a 
shared workspace and working hours, as is the case in an office 
environment. This results in a need to have tools to enable remote 
collaboration. Github solves the problem of sharing code across 
independent work-sessions, however when pair programming remotely, 
there is no adequate solutions other than sharing code over paste-bin 
and other such text sharing sites. Needless to say, these solutions 
leave much to be desired. An ideal solution would be an environment 
where collaborating programmers can actively make changes to source 
code that would be reflected in real-time, and can be discussed using 
an IM platform. Ideally, this functionality would be in a unified GUI 
application, so these activities can be done seamlessly. Thus arose the
need for CodeChat.

% Languages intro

When planning the project, we were faced with the choice of which
programming language(s) to use. We wanted to use a language with good
networking support and a good concurrency model. We also wanted a
language that is simple and intuitively clear, even if that meant that
the code was more verbose. These requirements led us to choose Go as
the primary language for the project. After the system was implemented,
we were curious how the developer and user experience would change if
implemented in a language with an opposing paradigm to Go: this led us
to try to implement select portions of the code in Python. This
exposure to multiple languages gave certain insights into how these two
languages relate to one another.

\section*{Architecture}

As this application requires connecting various users together, we 
figured that a client-server approach would be ideal. We also thought 
it would be best to implement the server with a modular and 
language-agnostic text protocol. While this may introduce added 
overhead in terms of performance, code size, and design/development 
time, the benefits of being able to create clients in a variety of 
languages, potentially implementing unique subsets of the server's 
commands, far outweigh any performance hits. The general architecture 
is a simple command protocol whereby a client can issue a command, and 
some optional payload, which the server then processes and writes back 
to all the other clients, as well as a status message to the issuing 
client.

\subsection*{Protocol}

This modularity allows the server to implement various commands and 
distribute the commands back to the connected clients. The clients, 
however can implement any particular subset of the server's commands, 
they only have to implement the bare minimum, connect and exit, to talk 
to the server. This means that while our server has the ability to 
support collaborative text editing, any client can simply implement a 
chat program with the same server, and ignore any unsupported commands 
from the server. Very cool.

To achieve language-agnosticism, we implemented a messaging protocol 
using JSON. We chose JSON because it is quickly (if not already) 
becoming a standard in server communications and most languages include 
libraries and packages for creating JSON objects. Additionally, since 
JSON is simply text, it can be very easily written to web 
sockets/connections in any language. I mentioned earlier the 
trade-offs, this is because when you implement a client and a server in 
the same language, you can take advantage of expected language features 
on both sides of the connection. This is especially the case in a 
language like Go, whereby communication across the network can 
essentially be the same as communication across threads (with 
known/expected data-types).

Currently implemented commands are: ``connect'', ``exit'',
``update-file'', ``message''.

\subsection*{Server}

The server is intended to be running on some network machine with a
publicly available IP address, it can additionally be run locally ---
but this inhibits the usefulness outside of debugging purposes. The
program establishes a network connection and loops listening for
incoming connections (as most servers do), when it receives a
connection, it spawns a separate ``goroutine'' (a lightweight thread)
to handle this connection's commands, then continues listening for more
connections.

Each connection is handled by a separate goroutine, and is encapsulated
in a Client data-type, which keeps track of each connected client's
username, connection handle, and the associated server. The goroutine
listens for incoming data on the connection, then proceeds to parse a
command from the JSON data. As it parses the command, it builds up an
IPC message type containing an error code, a response type to be
written back to this client, and a response type to be written back to
the other clients. When the command has been parsed, this IPC message
is then sent through a channel to the broadcast goroutine, which writes
back the appropriate JSON to each client.

Writing back to clients is handled in a single goroutine (broadcast)
which is established before any incoming connections are connected.
This thread reads on an internal channel, which is written to by each
client thread. When it gets an IPC message, it writes the return status
to the requesting client, and the response to the other clients.

\subsection*{Client}

The client essentially takes user input, serializes it into a JSON
command and writes it to the server. In the Go implementation, we were
able to encapsulate the client as a nice little library, exporting the
Connect(username, port), Write(command, payload), Read() functions, as
well as types for Client, and Messages. When combined with a GUI, the
program can simply pull user input from various parts of the interface,
and send them to the server through the client library. It then simply
needs to spawn a thread which reads from the server and updates the
appropriate parts of the interface. Care had to be taken when reading
from and writing to the interface, especially when operating in
different threads, since GTK is not thread-safe. GTK provides a set of
functions for protecting against data races and invalidated references,
\verb+gdk_threads_enter()+ and \verb+gdk_threads_exit()+.

\subsubsection*{GUI}

The graphics library we used was the Gimp ToolKit (GTK), we used 
version 2, because this was the version supported by the language 
bindings library for go, go-gtk, as well as the version supported by 
the language bindings library for Python, pygtk. This toolkit is very 
thorough and provides an excellent API and is fairly well documented. 
Go-gtk, however, is a thin wrapper for go around GTK's C API, and 
little documentation is provided, as it follows the API very closely. 
This was painful at times, because we had to delve into the source code 
in order to see which functionality from the original API was 
supported. We ended up actually contributing a few commits to the 
github for go-gtk in order to get some extra functionality we required. 
Pygtk is similarly a wrapper around the original C API, but provides 
better documentation. However, setting up the environment was more 
painful than for go. Due to the closeness of these bindings to the 
original API, converting the go-gtk code to pygtk was nearly as simple 
as converting CamelCase to snake\_case.

One part of the client we expected to be more difficult was the text 
editor, particularly implementing the editor-comforts such as 
syntax-highlighting, line numbers, monospaced fonts, and 
bracket-matching. To our pleasant surprise, GTK provides an excellent 
library for source-code editing interfaces, called GTKSourceView. GTK 
based Text-editors such as Gedit and Geany make use of this library, so 
in theory, our editor can eventually be as fully featured as these 
editors (even providing compatibility with their extensive plugins). 
This library made implementing these nice features as simple as making 
the correct GTK function calls (and implementing them in the go-gtk 
library).

\section*{Languages}

One of the first decisions we had to make in the project was what
language(s) to implement the project in. However, this was made easier
by the fact that we knew what we were planning on implementing, and
could see if languages would be suited to the task. Our first language
criteria was having a modern language. All members of our group are
very familiar with C/++, and wanted to use a language developed with
today's coding environment in mind. Next, we wanted to use a language
with lower overhead to improve running time. As the end-user experience
is largely interactive, responsiveness is a key consideration. We
decided that a language with poor performance would be unsuitable.
Additionally, we decided we needed a language with a good concurrency
model. As the server portion of CodeChat would be interfacing with
multiple clients, and each client would have portions operating
independently, the decision to have multi-threaded code came easy, and
we wanted a language with good native support for that. Finally, we
decided that we would prefer a language with intuitive clarity, even if
that came at the cost of brevity. Our reasoning here was that writing
100 lines of code in one hour was preferable to writing 10 lines of
code in 10 hours. Ultimately, these factors led us to choose Go as the
primary language. After we implemented the project, we decided to try
to implement the client in Python, a language we thought differed from
Go in several areas.


\subsection*{Go}

% Briefly introduce golang here

Go is an interesting new language developed at Google in 2009. It was 
originally touted as a new systems language to compete with C or C++, 
but the project has since avoided using this term. Go is statically 
typed, and ahead of time compiled directly to hardware. It also 
features a garbage collector, type safety, and a some dynamic-typing 
(type inference). As a modern language, it offers a robust standard 
library, excellent documentation, and an array of various 
language-support tools including the \verb+go-build+ build system, 
\verb+go-doc+ automatic documentation creation, and \verb+go-fmt+ 
automatic code formatter to the style guidelines. While it lacks a 
discrete package management system, such as Haskell's cabal or Node's 
npm, it provides a remote import tool, which, combined with the vast 
amount of projects on Github, is just as useful (if not more). 
Additionally, since go is a performance-oriented language in the modern 
computing world, and is geared towards scalable systems software, it 
provides an interesting concurrency model and IPC primitives.

The Go language is built around sound programming practices, and helps 
enforce or encourage these at any stage possible. The extensive guide 
``Effective Go" helps programmers understand and write `idiomatic Go.' 
Somewhat annoyingly, Go forces or encourages many of the sides fierce 
`style-debates' of C-style languages. For example, the debate on where 
to place block-starting curly braces is killed by forcing opening 
braces to be the last character on the previous line (if starting a new 
block). In this sense, the language has a much more restrictive syntax. 
The tabs-spaces debate is solved by the \verb+go-fmt+ tool, which 
automatically formats any go code to the language's preferred 
style-guide. Other stylistic choices and conventions are enforced by 
the language: any capitalized variables, functions, or structs are 
exported. There is no public or private keyword, simply a stylistic 
syntax defining whether or not something is visible outside the scope 
of the program. While these choices can seem irking and restrictive, 
they do serve an excellent goal --- ensuring a uniform code style 
across any program and facilitating both readability and coherence. It 
says something about the language that it would be difficult to have an 
``Obfuscated Go Contest." One of Go's language design goals was the 
principle that ``50 lines of clear, understandable, well documented 
code, is better than 10 lines of clever, confusing code." This isn't to 
say that go is verbose, but rather it is coherent and easily 
understandable.

One part of this philosophy is the enforcing of strict error checking. 
In the Go compilation process, any use of the language that would 
constitute a warning halts the compilation process. In this way, the Go 
compiler forces the user to conform to the clean style Go espouses. 
However, this feature is a double edged sword. Any identifier declared 
but not used constitutes a warning, and thus stops compilation. While 
this might not seem like a terrible feature, this rule also applies to 
libraries, meaning that library imports have to be modified when a 
function isn't used. The end result of this was frequent failed 
compilation when testing new features, as we would make declarations 
with the intention of implementing them at a later date. However, this 
stylistic enforcement helped in code organization by ensuring no stone 
was unturned in the implementation. This differed from the Python 
implementation, where we overlooked certain things in development, 
which would eventually cause debugging issues down the line. 

Stylistic points aside, go provides an excellent model for handling 
concurrency, which is best summed up by the phrase ``it is better to 
share memory by communicating, than to communicate by sharing memory." 
From this comes the idea of goroutines and channels. Goroutines, a pun 
on coroutine, are lightweight threads built in to the language. 
Channels are essentially safe, concurrent, blocking buffers which 
facilitate communication across goroutines. In the server, we used 
goroutines to handle each connection to the server, and a broadcasting 
goroutine which handles writing back to each connection. This ensures 
that reads on any given client only occurs in one thread, and writes to 
all clients only occur on a single thread.


\subsection*{Python}

% Briefly introduce python here

Python is an interpreted, high-level, and imperative programming 
language originally developed by Guido Van Rossum in 1991. Based partly 
on the ABC language, Python emphasises succinct, beautiful, simple 
code. Python is one of the more popular modern languages, with garbage 
collection, dynamic typing, and a robust community. Python code is 
generally highly abstracted, with the intended result being the 
programmer telling the machine what to do with little consideration to 
hardware or underlying systems, making the code very portable.

% Discuss language features here

In implementing the Python client, we used a number of key features of 
the Python programming language. Contrasting with the Go 
implementation, we made use of the object-oriented features of Python. 
This aided in code organization, as we were able to pass objects around 
as opposed to structs, and have member functions for handling them. One 
key problem we ran into while developing Python were concurrency issues 
with the native network library. The function to receive incoming 
network messages is blocking and buffered, leading to messages being 
grouped by the receiver, and not being responsive. Furthermore, making 
the function non-blocking caused the system resources to be hogged by 
the receiver. Because of this, we were able to implement only writing 
to the network, not reading from it. We suspect using the 
\verb+asyncore+ library would solve most of these issues with 
asynchronous network reads.

\subsection*{Comparison}

Following our completion of the client in Go, we decided to try to 
implement the client in Python as well, to see how a language designed 
with brevity in mind compares to one that partially eschews it for 
further simplicity and less abstraction. In doing so, we noticed 
several key differences in the languages we used. One such difference 
was in the debugging process, specifically in the ease in debugging Go 
code. Since Go code is compiled, with harsh requirements, static code 
analysis is done, making catching certain types of errors very easy. In 
contrast, Python is interpreted, with no static analysis being done. 
This resulted in lots of development difficulties, as errors in the 
code had to be hunted down very methodically. Another key difference in 
the languages was the difference in code length. The client written in 
Go ended up being around 490 %Put correct number here lines of code, 
while the Python client ended up being around 270  %Put correct number 
here lines of code. These differences are consistent with our 
expectations regarding development, with the Python living up to it's 
expectations in regards to length. One major difference we found 
between Python and Go was the concurrency model used by the language. 
Python has simple thread support through the use of a library. However, 
it features few useful concurrency primitives. In contrast, the 
concurrency model built into the Go language was robust and powerful, 
making multi-threaded applications easy to develop. Unfortunately, 
because we were unable to complete the Python implementation, we have 
an incomplete analysis of the differences between Python and Go.

\section*{Testing}

Simple stress tests performed on the server yielded around 
$\unitfrac[100]{messages}{second}$ across a network, and around 
$\unitfrac[500]{messages}{second}$ on a local instance. These tests 
simply monitored the ability of the server to handle a single command 
and write back to the client. No other clients were connected to the 
server. Ideally the tests would be written to be more comprehensive to 
accommodate for a variety of different parameters:

\begin{itemize}
	\item Length of payload (message or file)
	\item Number of idle clients connected to the server
	\item Number of active clients connected to the server
\end{itemize}

Thankfully the nice client module we provided in go (abstracting the 
client-server communications from the GUI client), provide an excellent 
platform for writing this test suite. The results of these tests can be 
used to identify weak points in both the client library and the server 
code.

For now, we are satisfied with the current performance of the server, 
but in the future, we would love the opportunity to write a test suite 
and use it to fully optimize the performance.

\section*{For the future...}

We fully intend to take this product and develop it into a piece of 
software we would use on a regular basis. An area of improvement that 
is obvious right off the bat is increasing the granularity on the 
updates of the code-editor. In the current implementation, the entire 
file-buffer is yanked and sent to the server. While this may seem 
inefficient, it actually solved our multiple-writers problem, which was 
one of our original main concerns. Since every time a client receives a 
new file, it simply replaces the contents of the file-buffer with the 
new file, we never have the problem of having more than one writer. 
This, of course is both inefficient and doesn't optimize collaboration, 
since every time the file is updated, the users cursor position is lost. 
This problem can easily be solved by fine-graining the updates, for 
example, sending a single line at a time, whenever it is modified. 
While increasing efficiency, it also increases complexity and enhanced 
server and client logic flow to handle updating files.

\end{document}
